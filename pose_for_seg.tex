\documentclass[10pt,twocolumn,letterpaper]{article}

\usepackage{cvpr}
\usepackage{times}
\usepackage{epsfig}
\usepackage{graphicx}
\usepackage{amsmath}
\usepackage{amssymb}
\usepackage{amsfonts}       % blackboard math symbols
\usepackage{nicefrac}       % compact symbols for 1/2, etc.
\usepackage{microtype}      % microtypography

\usepackage{url}
\usepackage[table]{xcolor}
\usepackage{bbm}
\usepackage{booktabs}
\usepackage[T1]{fontenc}
\usepackage{fix-cm}
\usepackage{array}
\usepackage{epsfig}
%\usepackage{mathabx}
\usepackage{dsfont}
\usepackage{multirow}


\usepackage{times}
\usepackage{helvet}
\usepackage{courier}
\usepackage{graphicx}
\usepackage{bm}
\usepackage{amsmath,amssymb} % define this before the line numbering.
\usepackage{color}
%\usepackage[width=122mm,left=12mm,paperwidth=146mm,height=193mm,top=12mm,paperheight=217mm]{geometry}
\usepackage{bbm}
\usepackage{epstopdf}
\usepackage{caption}
\usepackage{subcaption}
\usepackage{enumitem}
\usepackage{calc}
\usepackage{multirow}
\usepackage{xspace}
\usepackage{booktabs}
\usepackage{mathrsfs}
\usepackage{array}

% Include other packages here, before hyperref.

% If you comment hyperref and then uncomment it, you should delete
% egpaper.aux before re-running latex.  (Or just hit 'q' on the first latex
% run, let it finish, and you should be clear).
\usepackage[pagebackref=true,breaklinks=true,letterpaper=true,colorlinks,bookmarks=false]{hyperref}

% \cvprfinalcopy % *** Uncomment this line for the final submission

\def\cvprPaperID{936} % *** Enter the CVPR Paper ID here
\def\httilde{\mbox{\tt\raisebox{-.5ex}{\symbol{126}}}}

% Pages are numbered in submission mode, and unnumbered in camera-ready
\ifcvprfinal\pagestyle{empty}\fi
\begin{document}

%%%%%%%%% TITLE
\title{DeepLocSeg: Camera Pose Estimation and Scene Parsing with 3D Semantic World and Deep CNN}

\author{First Author\\
Institution1\\
Institution1 address\\
{\tt\small firstauthor@i1.org}
% For a paper whose authors are all at the same institution,
% omit the following lines up until the closing ``}''.
% Additional authors and addresses can be added with ``\and'',
% just like the second author.
% To save space, use either the email address or home page, not both
\and
Second Author\\
Institution2\\
First line of institution2 address\\
{\tt\small secondauthor@i2.org}
}

\maketitle
%\thispagestyle{empty}

%%%%%%%%% ABSTRACT
\begin{abstract}
Visual-based outdoor navigation requires accurately localizing the camera and preferably per-pixel semantic understanding. It can be widely applied for robotic navigation.
However, system solely relying on visual signal is non-robust due to visual confusion across multiple scenes. 
Thus, coarse signals from motion sensors, \eg GPS and IMU, are usually considered as a localization prior~\cite{}.
In this paper, we propose a practical deep learning based method for localizing the camera and parsing the recorded video simultaneously in a single framework, dependent on visual and motion sensors.
Assuming we have a 3D semantic world, and the testing video is recording inside. Rough pose signal is also obtained online. In our system, for each frame in a moving camera, based on the obtained coarse pose signal, we render a label map out from our 3D world online, and feed it to a pose CNN jointly with the current frame of image, yielding a corrected pose. 
Then, a multi-layer recurrent neural network (RNN) is performed afterwards in order to capture high-order temporal information. 
Finally, a new label map is rendered again based on the corrected pose, which is feed into a segment CNN combining with the image.
In order to perform the experiments,  we built a dataset with a semantic labeled real world 3D map, jointly with many recorded videos with ground truth pose from high accurate motion sensors. We show that firstly, in a relative large environment, unlike the PoseNet~\cite{}, it is important to have a reference coarse pose signal. In addition, semantic information and pose are mutually beneficial in learning more robust networks. Finally, various ablation studies are performed, which demonstrate the effectiveness of the proposed system.
\end{abstract}

%%%%%%%%% BODY TEXT
\section{Introduction}
% the problem we are solving, distinguish with previous problems
Estimating the pose of a calibrated camera given an video and 



\section{Related work}
% traditional key points, slam etc
% marc's paper
% hongdong's paper
% posenet related
% iccv kalman 
% domain transfer etc


\section{Data collection}
\begin{table*}[t]
\centering
\caption{Compare our data with the other related datasets.}
\label{tbl:data}
\fontsize{6.5}{7}\selectfont
\bgroup
\def\arraystretch{1.4}
\begin{tabular}{lllllllllll}
\thickhline
\multirow{2}{*}{Dataset} & \multirow{2}{*}{Test data} & \multicolumn{2}{l}{Supervision} & \multicolumn{4}{l}{Lower the better} & \multicolumn{3}{l}{Higher the better}               \\ 
\cline{3-11} 
                                                             &                                                   & Depth          & Pose           & Abs Rel  & Sq Rel & RMSE  & RMSE log & $\delta < 1.25$ & $\delta<1.25^2$ & $\delta<1.25^3$ \\ \hline
\multicolumn{1}{l|}{Train set mean}                          & \multicolumn{1}{l|}{\multirow{8}{*}{Eigen split}} & \checkmark     &                & 0.403    & 5.530  & 8.709 & 0.403    & 0.593           & 0.776           & 0.878           \\
\multicolumn{1}{l|}{Coarse}            & \multicolumn{1}{l|}{}                             & \checkmark     &                & 0.214    & 1.605  & 6.563 & 0.292    & 0.673           & 0.884           & 0.957           \\
\multicolumn{1}{l|}{Fine}               & \multicolumn{1}{l|}{}                             & \checkmark     &                & 0.203    & 1.548  & 6.307 & 0.282    & 0.702           & 0.890           & 0.958           \\
\multicolumn{1}{l|}{supervised}   & \multicolumn{1}{l|}{}                             & \checkmark     &                & 0.122    & 0.763  & 4.815 & 0.194    & 0.845           & 0.957           & 0.987           \\
\multicolumn{1}{l|}{\cite{kuznietsov2017semi} unsupervised} & \multicolumn{1}{l|}{}                             &                & \checkmark     & 0.308    & 9.367  & 8.700 & 0.367    & 0.752           & 0.904           & 0.952           \\
\multicolumn{1}{l|}{\cite{godard2016unsupervised}}                  & \multicolumn{1}{l|}{}                             &                & \checkmark     & 0.148    & 1.344  & 5.927 & 0.247    & 0.803           & 0.922           & 0.964           \\
\multicolumn{1}{l|}{\cite{zhou2017unsupervised}}                    & \multicolumn{1}{l|}{}                             &                &                & 0.208    & 1.768  & 6.856 & 0.283    & 0.678           & 0.885           & 0.957           \\
\multicolumn{1}{l|}{Ours}                                    & \multicolumn{1}{l|}{}                             &                &                & 0.182    & 1.481  & 6.501 & 0.267    & 0.725           & 0.906           & 0.963           \\ \hline
\multicolumn{1}{l|}{Train set mean}                          & \multicolumn{1}{r|}{\multirow{2}{*}{}}            & \checkmark     &                & 0.398    & 5.519  & 8.632 & 0.405    & 0.587           & 0.764           & 0.880           \\
\multicolumn{1}{l|}{\cite{godard2016unsupervised}}                  & \multicolumn{1}{r|}{}                             &                & \checkmark     & 0.124    & 1.388  & 6.125 & 0.217    & 0.841           & 0.936           & 0.975           \\
\multicolumn{1}{l|}{\cite{Vijayanarasimhan17}}        & \multicolumn{1}{l|}{KITTI split}                  &                &                & -        & -      & -     & 0.340    & -               & -               & -               \\
\multicolumn{1}{l|}{\cite{zhou2017unsupervised}}                    & \multicolumn{1}{l|}{\multirow{2}{*}{}}            &                &                & 0.216    & 2.255  & 7.422 & 0.299    & 0.686           & 0.873           & 0.951           \\
\multicolumn{1}{l|}{Ours}                                    & \multicolumn{1}{l|}{}                             &                &                & 0.1648   & 1.360  & 6.641 & 0.248    & 0.750           & 0.914           & 0.969           \\ \hline
\end{tabular}
\egroup
\vspace{-0.3\baselineskip}
\end{table*}

\section{Localize and parsing in the virtual}

\section{Experiments}

\paragraph{Implementation details.}

\paragraph{Quantitative evaluation.}

\paragraph{Qualitative results.}

\paragraph{Discussion.}


\section{Conclusion}



%-------------------------------------------------------------------------
\begin{figure}[t]
\begin{center}
\fbox{\rule{0pt}{2in} \rule{0.9\linewidth}{0pt}}
   %\includegraphics[width=0.8\linewidth]{egfigure.eps}
\end{center}
   \caption{Example of caption.  It is set in Roman so that mathematics
   (always set in Roman: $B \sin A = A \sin B$) may be included without an
   ugly clash.}
\label{fig:long}
\label{fig:onecol}
\end{figure}

\subsection{Miscellaneous}

\noindent
Compare the following:\\
\begin{tabular}{ll}
 \verb'$conf_a$' &  $conf_a$ \\
 \verb'$\mathit{conf}_a$' & $\mathit{conf}_a$
\end{tabular}\\
See The \TeX book, p165.

The space after \eg, meaning ``for example'', should not be a
sentence-ending space. So \eg is correct, {\em e.g.} is not.  The provided
\verb'\eg' macro takes care of this.

When citing a multi-author paper, you may save space by using ``et alia'',
shortened to ``\etal'' (not ``{\em et.\ al.}'' as ``{\em et}'' is a complete word.)
However, use it only when there are three or more authors.  Thus, the
following is correct: ``
   Frobnication has been trendy lately.
   It was introduced by Alpher~\cite{Alpher02}, and subsequently developed by
   Alpher and Fotheringham-Smythe~\cite{Alpher03}, and Alpher \etal~\cite{Alpher04}.''

This is incorrect: ``... subsequently developed by Alpher \etal~\cite{Alpher03} ...''
because reference~\cite{Alpher03} has just two authors.  If you use the
\verb'\etal' macro provided, then you need not worry about double periods
when used at the end of a sentence as in Alpher \etal.

For this citation style, keep multiple citations in numerical (not
chronological) order, so prefer \cite{Alpher03,Alpher02,Authors14} to
\cite{Alpher02,Alpher03,Authors14}.


\begin{figure*}
\begin{center}
\fbox{\rule{0pt}{2in} \rule{.9\linewidth}{0pt}}
\end{center}
   \caption{Example of a short caption, which should be centered.}
\label{fig:short}
\end{figure*}

%------------------------------------------------------------------------
\section{Formatting your paper}

All text must be in a two-column format. The total allowable width of the
text area is $6\frac78$ inches (17.5 cm) wide by $8\frac78$ inches (22.54
cm) high. Columns are to be $3\frac14$ inches (8.25 cm) wide, with a
$\frac{5}{16}$ inch (0.8 cm) space between them. The main title (on the
first page) should begin 1.0 inch (2.54 cm) from the top edge of the
page. The second and following pages should begin 1.0 inch (2.54 cm) from
the top edge. On all pages, the bottom margin should be 1-1/8 inches (2.86
cm) from the bottom edge of the page for $8.5 \times 11$-inch paper; for A4
paper, approximately 1-5/8 inches (4.13 cm) from the bottom edge of the
page.

%-------------------------------------------------------------------------
\subsection{Margins and page numbering}

All printed material, including text, illustrations, and charts, must be kept
within a print area 6-7/8 inches (17.5 cm) wide by 8-7/8 inches (22.54 cm)
high.



%-------------------------------------------------------------------------
\subsection{Type-style and fonts}

Wherever Times is specified, Times Roman may also be used. If neither is
available on your word processor, please use the font closest in
appearance to Times to which you have access.

MAIN TITLE. Center the title 1-3/8 inches (3.49 cm) from the top edge of
the first page. The title should be in Times 14-point, boldface type.
Capitalize the first letter of nouns, pronouns, verbs, adjectives, and
adverbs; do not capitalize articles, coordinate conjunctions, or
prepositions (unless the title begins with such a word). Leave two blank
lines after the title.

AUTHOR NAME(s) and AFFILIATION(s) are to be centered beneath the title
and printed in Times 12-point, non-boldface type. This information is to
be followed by two blank lines.

The ABSTRACT and MAIN TEXT are to be in a two-column format.

MAIN TEXT. Type main text in 10-point Times, single-spaced. Do NOT use
double-spacing. All paragraphs should be indented 1 pica (approx. 1/6
inch or 0.422 cm). Make sure your text is fully justified---that is,
flush left and flush right. Please do not place any additional blank
lines between paragraphs.

Figure and table captions should be 9-point Roman type as in
Figures~\ref{fig:onecol} and~\ref{fig:short}.  Short captions should be centred.

\noindent Callouts should be 9-point Helvetica, non-boldface type.
Initially capitalize only the first word of section titles and first-,
second-, and third-order headings.

FIRST-ORDER HEADINGS. (For example, {\large \bf 1. Introduction})
should be Times 12-point boldface, initially capitalized, flush left,
with one blank line before, and one blank line after.

SECOND-ORDER HEADINGS. (For example, { \bf 1.1. Database elements})
should be Times 11-point boldface, initially capitalized, flush left,
with one blank line before, and one after. If you require a third-order
heading (we discourage it), use 10-point Times, boldface, initially
capitalized, flush left, preceded by one blank line, followed by a period
and your text on the same line.

%-------------------------------------------------------------------------
\subsection{Footnotes}

Please use footnotes\footnote {This is what a footnote looks like.  It
often distracts the reader from the main flow of the argument.} sparingly.
Indeed, try to avoid footnotes altogether and include necessary peripheral
observations in
the text (within parentheses, if you prefer, as in this sentence).  If you
wish to use a footnote, place it at the bottom of the column on the page on
which it is referenced. Use Times 8-point type, single-spaced.


%-------------------------------------------------------------------------
\subsection{References}

List and number all bibliographical references in 9-point Times,
single-spaced, at the end of your paper. When referenced in the text,
enclose the citation number in square brackets, for
example~\cite{Authors14}.  Where appropriate, include the name(s) of
editors of referenced books.

\begin{table}
\begin{center}
\begin{tabular}{|l|c|}
\hline
Method & Frobnability \\
\hline\hline
Theirs & Frumpy \\
Yours & Frobbly \\
Ours & Makes one's heart Frob\\
\hline
\end{tabular}
\end{center}
\caption{Results.   Ours is better.}
\end{table}

{\small
\bibliographystyle{ieee}
\bibliography{egbib}
}

\end{document}
