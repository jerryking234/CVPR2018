\section{Dataset Preparation}
\label{sec:data_collection}
\begin{table*}[t]
\center
\fontsize{8}{9}\selectfont
\begin{tabular}{lcccc}
\toprule[0.2 em]
% \thickhline
Dataset & Real data & Camera pose & 3D semantic map & Video per-frame labeling   \\
\hline
\multicolumn{1}{l|}{CamVid~\cite{brostow2009semantic}}     &\checkmark        & -              & -              &  -   \\
\multicolumn{1}{l|}{KITTI~\cite{geiger2012we}}      &\checkmark  & \checkmark     & sparse points  & -   \\
\multicolumn{1}{l|}{CityScapes~\cite{Cordts2016Cityscapes}} &\checkmark  & -              &  -             & selected frames  \\
\multicolumn{1}{l|}{Toronto~\cite{wang2016torontocity}}    &\checkmark  & \checkmark     & 3d building $\&$ road & selected pixels \\
\hline
\multicolumn{1}{l|}{Synthia~\cite{RosCVPR16}}    & -          & \checkmark     & -       &\checkmark     \\
\multicolumn{1}{l|}{Play for benchmark~\cite{richter2017playing}} &-   & \checkmark     & -     &\checkmark  \\
\hline
\multicolumn{1}{l|}{Ours}              & \checkmark &\checkmark    &dense point cloud  & \checkmark    \\
\toprule[0.2 em]
\end{tabular}
\caption{Compare our dataset with the other related outdoor street-view datasets for our task. 'Real data' mean whether the data is collected from our physical world.
'3D semantic map' means whether it contains a 3D map of scenes with object semantic label. 'Per-pixel label' means whether it has per-pixel semantic label.}
 %'Temporal variations' mean whether the recorded video can roughly cover the whole scene, but have multiple.
%\textcolor[rgb]{1.00,0.00,0.00}{we don't have temporal varations here, we didn't present results on this point neither, that last column needs to be removed. I would also argue the Toronto one is good.
%it has 3D model, so it is dense. In the scope of this paper, we want to paint the picture that 3D maps will be available, so we don't want to over-sell the  uniqueness of our data set.  }}
\label{tbl:data}
\vspace{-0.3\baselineskip}
\end{table*}

\paragraph{Motivation.}
As described in the \secref{sub:framework}, our system is design to work with available motion sensors and a semantic 3D map.
However, similar outdoor datasets such as KITTI and CityScapes do not containing such information, in particular the 3D map. The Tortoro City dataset~\cite{wang2016torontocity} could be used. However, it is not open yet. As summarized in \tabref{tbl:data}, we list several key properties to perform our experiments. There is no public data set that can be fully used. 
%As can be seen, existing ones that are already open do not satisfy our requirements, especially for pose estimation,
% \textcolor[rgb]{1.00,0.00,0.00}{as I explained in the table, I suggest that we drop this argument, we need to have a 3D semantic map, none has it. the temporal effect needs to be tuned down, otherwise, it will be viewed as too narraow}
% for training, we want the recorded video to roughly cover the 3D environment, so that when new images come in, appearance similar views had been seen during training. This means repeated recording at similar spatial locations are required in the data, which we call temporal variations in \tabref{tbl:data}.
% However, most existing datasets are a temporal snapshot at some locations in the environment, which requires us to collect a new dataset ourselves.

\paragraph{Data collection.}
We use a mobile LIDAR scanner from $Riegl$ to collect point clouds of the static 3D map with high granularity. As shown in \figref{fig:data}(a). The captured point cloud density is much higher than the Velodyne\footnote{http://www.velodynelidar.com/} used by KITTI~\cite{geiger2012we}.
% Amazingly, it is even able to capture the hight changes of curb on road.
However, different from the sparse Velodyne LIDAR, our mobile scanner utilizes a single laser beam to scan a vertical circle. As the scanner moves, it is able to scan its surroundings as a push-broom camera. It cannot correctly capture moving objects, they will be compressed, expanded, or completely missing in the resulting point cloud.
In order to remove those fake object points (circled in blue at \figref{fig:data}(b)), thanks to the fact that the 3D world is recorded multiple rounds, and each round has a full coverage of the 3D map, we can use point cloud consistency to handle the issue.
Specifically, we register the 3D point clouds from different rounds. From the merged point cloud, the points with high temporal consistency are kept, while those with low consistency are removed. Formally, the condition to keep a point $\ve{x}$ in round $j$ is,
\begin{align}
\sum_{i=0}^{r}{\mathbbm{1}(\exists~\|\ve{x}_i - \ve{x}_j\| < \epsilon_d )} / r \geq \delta
\end{align}
where $\delta = 0.7$ and $\epsilon_d = 0.025$ in our experiments, and $\mathbb{1}()$ is a indicator function. Notice due to the symmetric property of $\ve{x}_i$ and $\ve{x}_j$, we obtain the static points for all rounds by looping over the points of a single round. We keep the merged point cloud as a static background $\hua{M}$ for further labelling.

For video, we use two front-facing cameras with a resolution of \by{2048}{2432}, and both cameras are calibrated \wrt the LIDAR. All the images from videos are undistorted, therefore a single projective transform can link a 3D point with a 2D pixel. .
% talk about registration and removing moving objects inside data

\begin{figure}[!htpb]
\begin{center}
\includegraphics[width=\linewidth]{fig/dataset.pdf}
\end{center}
   \caption{An example of our collected street-view dataset. (a) Image. (b) Rendered label map with 3D points, with invalidn object points (circled in blue). (c)Rendered label map with 3D points after merge. (d) $\&$ (e) Rendered depth map of background and rendered label map with class dependent splatting from 3D points (\secref{sub:render}). (e) Merge label map with missing region in-painted, labelled object and sky.}
\label{fig:data}
\end{figure}

% Undistortion of the image
\paragraph{Label 3D map and videos.}
% talk about labelling in 3D
In order to have semantic labelling of each frame in video, we handle static background, static objects and moving objects separately.
Firstly, for static background, we directly perform labelling on the 3D point cloud $\hua{M}$ which is then projected to images, yielding labelled background for all the frames.
Specifically, we over-segment the point cloud data into point clusters based on spatial distance and normal directions, and then we ask labellers to label each cluster of points manually.
For static objects in each round of video, we prune out the points of static background, and label the remaining points of objects. Due to the fact that when object is static, \eg parked cars, the shape and location of captured points will be highly recognizable, whereas when the object is moving on the road, the points will be fuzzy, we ask labeller to label those can be well recognized as static objects. Last, after projecting the 3D to 2D, only moving objects remain to be labeled. Here, we adopt an active labelling strategy, by first training an object segmentation model using a SOTA algorithm~\cite{WuSH16e}, and ask labellers to correct and label the masks of moving objects.

Finally, as shown in \figref{fig:data}(c), the projected label from 3D points is not perfect, due to missing points either too far away or reflection. We handle such issue by using splatting techniques in computer graphics, that is to turn each point into a small square as discussed in \secref{sub:render} (\figref{fig:data}(d)), and then ask labeller to fix missing regions, yielding the final label map (\figref{fig:data}(e)).
With such a strategy, labelling efficiency of video can be vastly increased, avoiding labelling edge rich regions like trees and poles on the street, especially when occlusion is happened.
We provide the labelled video in our supplementary materials.

